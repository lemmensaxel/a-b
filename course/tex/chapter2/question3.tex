\begin{question}
Bewijs dat een eindige toestandsautomaat altijd een taal herkent die door een reguliere expressie wordt beschreven. Doe dat door een eindige toestandsautomaat te transformeren naar een gegeneraliseerde niet-deterministische eindige toestandsautomaat met slechts twee toestanden.
\end{question}

Voor we hier aan beginnen defini\"eren we de \emph{GNFA} als volgt:
\begin{theorem}[GNFA]
Een GNFA (gegeneraliseerde niet-deterministische eindige automaat) is een eindige toestandsmachine met de volgende wijzigingen en beperkingen:
\begin{enumerate}
\item Er is slechts 1 eindtoestand $\neq$ starttoestand
\item Er is juist 1 boog van de starttoestand naar elke andere toestand en er komen geen pijlen aan (buiten de startpijl)
\item Er is juist 1 boog van elke toestand naar de eindtoestand, maar er vertrekken geen pijlen vanuit de eindtoestand
\item Tussen elke andere 2 toestanden is er juist 1 boog in beide richtingen
\item Er is juist 1 boog van elke andere toestand naar zichzelf
\item De bogen hebben als label een reguliere expressie
\end{enumerate}
\end{theorem}

In wat volgt bewijzen we dat een eindige toestandsautomaat altijd een taal herkent die door een reguliere expressie wordt beschreven. We volgen volgend stappenplan:
$$ \text{NFA} \rightarrow \text{GNFA} \rightarrow \text{GNFA met 2 toestanden} \rightarrow \text{reguliere expressie} $$
In de opgave spreken ze van een "eindige toestandsautomaat", we gaan er hier echter vanuit dat we starten vanaf een NFA. Dit mag vermits elke DFA omgezet kan worden naar een NFA.

\begin{proof}
\subsubsection*{(a) NFA $\rightarrow$ GNFA}
We kunnen onze NFA omzetten naar een GNFA als volgt:
\begin{enumerate}
\item Maak een nieuwe starttoestand en een nieuwe (unieke) eindtoestand
\item Teken $\epsilon$-bogen van de nieuwe begintoestand naar de oude begintoestand en van elke oude eindtoestand naar de nieuwe eindtoestand.
\item Teken ontbrekende bogen met een $\phi$.
\item Als er tussen twee toestanden twee of meer parallele gerichte bogen zijn, neem die dan samen met als label de unie van de labels van de parallelle bogen.
\end{enumerate}
Nu we een GNFA geconstrueerd hebben moeten we deze reduceren naar een GNFA met 2 toestanden.

\subsubsection*{(b) Reduceren van de GNFA}
Dit doen we door herhaaldelijk een willekeurige toestand X verschillend van de start- of eindtoestand te kiezen en deze knoop de verwijderen als volgt:
\begin{enumerate}
\item Kies toestanden A en B zodat er bogen zijn van: \todo{Figuur?}
\begin{itemize}
\item A naar B met label $E_4$
\item A naar X met label $E_1$
\item X naar zichzelf met label $E_2$
\item X naar B met label $E_3$
\end{itemize}
\item Vervang het label op de boog van A naar B door $E_4 | E_1 E_2^* E_3$.
\item Doe dit voor alle koppels A en B
\item Verwijder de knoop X met alle bogen die erin toekomen of vertrekken.
\end{enumerate}
Indien er geen toestanden meer zijn, ga naar stap (c).

\subsubsection*{(c) Bepaal RE}
De GNFA heeft nu exact 2 toestanden (een start- en eindtoestand) met daartussen 1 boog. Deze boog heeft nu een reguliere expressie als label. Dit is de reguliere expressie die dezelfde taal bepaald als de oorspronkelijke NFA.
\end{proof}
